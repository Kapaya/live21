%%
%% This is file `sample-sigconf.tex',
%% generated with the docstrip utility.
%%
%% The original source files were:
%%
%% samples.dtx  (with options: `sigconf')
%%
%% IMPORTANT NOTICE:
%%
%% For the copyright see the source file.
%%
%% Any modified versions of this file must be renamed
%% with new filenames distinct from sample-sigconf.tex.
%%
%% For distribution of the original source see the terms
%% for copying and modification in the file samples.dtx.
%%
%% This generated file may be distributed as long as the
%% original source files, as listed above, are part of the
%% same distribution. (The sources need not necessarily be
%% in the same archive or directory.)
%%
%% The first command in your LaTeX source must be the \documentclass command.
\documentclass[sigconf,10pt]{acmart}

\settopmatter{printfolios=true}

\usepackage{hyperref}


\providecommand{\tightlist}{%
  \setlength{\itemsep}{0pt}\setlength{\parskip}{0pt}}

%%
%% \BibTeX command to typeset BibTeX logo in the docs
\AtBeginDocument{%
  \providecommand\BibTeX{{%
    \normalfont B\kern-0.5em{\scshape i\kern-0.25em b}\kern-0.8em\TeX}}}

%% Rights management information.  This information is sent to you
%% when you complete the rights form.  These commands have SAMPLE
%% values in them; it is your responsibility as an author to replace
%% the commands and values with those provided to you when you
%% complete the rights form.


%%
%% Submission ID.
%% Use this when submitting an article to a sponsored event. You'll
%% receive a unique submission ID from the organizers
%% of the event, and this ID should be used as the parameter to this command.
%%\acmSubmissionID{123-A56-BU3}

%%
%% The majority of ACM publications use numbered citations and
%% references.  The command \citestyle{authoryear} switches to the
%% "author year" style.
%%
%% If you are preparing content for an event
%% sponsored by ACM SIGGRAPH, you must use the "author year" style of
%% citations and references.
%% Uncommenting
%% the next command will enable that style.
%%\citestyle{acmauthoryear}
\pagenumbering{gobble}
%%
%% end of the preamble, start of the body of the document source.
\begin{document}

%%
%% The "title" command has an optional parameter,
%% allowing the author to define a "short title" to be used in page headers.
\title{Joker: A Unified Interaction Model For Web Customization}

%%
%% The "author" command and its associated commands are used to define
%% the authors and their affiliations.
%% Of note is the shared affiliation of the first two authors, and the
%% "authornote" and "authornotemark" commands
%% used to denote shared contribution to the research.

%\author{Kapaya Katongo}
%\affiliation{%
%  \institution{MIT CSAIL}
%  \city{Cambridge, MA}
%  \country{USA}
%}
%\email{kkatongo@mit.edu}

%\author{Geoffrey Litt}
%\affiliation{%
%  \institution{MIT CSAIL}
%  \city{Cambridge, MA}
%  \country{USA}
%}
%\email{glitt@mit.edu}

%\author{Kathryn Jin}
%\affiliation{%
%  \institution{MIT CSAIL}
%  \city{Cambridge, MA}
%  \country{USA}
%}
%\email{kjin@mit.edu}

%\author{Daniel Jackson}
%\affiliation{%
% \institution{MIT CSAIL}
%  \city{Cambridge, MA}
%  \country{USA}
%}
%\email{dnj@csail.mit.edu}

%%
%% By default, the full list of authors will be used in the page
%% headers. Often, this list is too long, and will overlap
%% other information printed in the page headers. This command allows
%% the author to define a more concise list
%% of authors' names for this purpose.
% \renewcommand{\shortauthors}{Trovato and Tobin, et al.}

%%
%% The abstract is a short summary of the work to be presented in the
%% article.
%% \begin{abstract}
%%  
%% \end{abstract}

%%
%% The code below is generated by the tool at http://dl.acm.org/ccs.cfm.
%% Please copy and paste the code instead of the example below.
%%
%% From HERE
%%\begin{CCSXML}
%%<ccs2012>
%%<concept>
%%<concept_id>10011007.10011006.10011066.10011069</concept_id>
%%<concept_desc>Software and its engineering~Integrated and visual development environments</concept_desc>
%%<concept_significance>500</concept_significance>
%%</concept>
%%</ccs2012>
%%\end{CCSXML}

%% \ccsdesc[500]{Software and its engineering~Integrated and visual development environments}
% To HERE

%%
%% Keywords. The author(s) should pick words that accurately describe
%% the work being presented. Separate the keywords with commas.
%% \keywords{end-user programming, software customization, web scraping, programming-by-example, program synthesis}

%% A "teaser" image appears between the author and affiliation
%% information and the body of the document, and typically spans the
%% page.
% \begin{teaserfigure}
%  \includegraphics[width=\textwidth]{sampleteaser}
%  \caption{Seattle Mariners at Spring Training, 2010.}
%  \Description{Enjoying the baseball game from the third-base
%  seats. Ichiro Suzuki preparing to bat.}
%  \label{fig:teaser}
%\end{teaserfigure}

%%
%% This command processes the author and affiliation and title
%% information and builds the first part of the formatted document.
\maketitle

\hypertarget{sec:introduction}{%
\section{Introduction}\label{sec:introduction}}

Many websites do not meet the exact needs of all of their users, so
millions of people use browser extensions and userscripts
\citep{zotero-224, 2021f} to customize them. However, these tools only
allow end-users to install customizations built by programmers. End-user
web customization systems like Sifter \citep{huynh2006}, Vegemite
\citep{lin2009} and Wildcard \citep{litt2020} provide a more accessible
approach, allowing anyone to create bespoke customizations without
performing traditional programming.

These tools each provide different useful mechanisms for end-user
customization, but they share a common design limitation: they have a
rigid separation between the two stages of the web customization
process. First, in the \emph{extraction} or \emph{scraping} phase, users
get data from the website into a structured, tabular format. Second, in
the \emph{augmentation} phase, users perform augmentations like adding
new columns derived from the data, or sorting the data. For example, in
Vegemite, a user can extract a list of addresses from a housing catalog,
and then augment the data by computing a walkability score for each
address.

This separation between extraction and augmentation poses an important
barrier to usability. A user study \citep{lin2009} of Vegemite wrote
that ``it was confusing to use one technique to create the initial
table, and another technique to add information to a new column.'' The
creators of Sifter similarly reported \citep{huynh2006} that ``the
necessity for extracting data before augmentation could take place was
poorly understood, if understood at all.'' In Wildcard, end-users cannot
augment a website at all until a programmer has written and shared
extraction code for that website in Javascript \citep{litt2020}. These
tools all impose a sequential workflow in which users must first extract
all the data they need, and then perform all their desired
augmentations. This workflow exemplifies the more general problem in
interface design of forcing users to make premature commitments to
formal structure \citep{shipman1999, blackwell2001}.

In this paper, we present a new approach to web customization that
combines extraction and augmentation in a unified interaction model. Our
key idea is to develop a domain specific language (DSL) that encompasses
both extraction and augmentation tasks, along with a
programming-by-demonstration (PBD) interface that makes it easy for
end-users to program in the language. This unified interaction model
allows end-users to move seamlessly between extraction and augmentation,
resulting in a more iterative and free-form workflow for web
customization.

To demonstrate and evaluate this model, we have built a browser
extension called Joker, an extension of the Wildcard customization tool.
The original Wildcard system \citep{litt2020} adds a spreadsheet-like
table to a website and establishes a bidirectional synchronization
between the website and the table. This allows users to customize a
website, by filtering and sorting page elements, and adding user
annotations, derived values and calls to web services. Although Wildcard
offers a declarative formula language for augmenting the page, a
conventional imperative language (namely JavaScript) is used for the
extraction step, and the extraction code cannot be modified during
augmentation.

Joker makes two primary contributions:

\textbf{A unified formula language for extraction \& augmentation}:
Wildcard's formula language only supported primitive values like strings
and numbers aimed at augmentation. Joker extends this language by
introducing Document Object Model (DOM) elements as a new type of value,
and adding a new set of formulas for performing operations on them. This
includes querying elements with Cascading Style Sheets (CSS) selectors
and traversing the DOM. With this approach, a single formula language is
used to express both extraction and augmentation tasks, even within a
single formula expression.

\textbf{A PBD interface for creating extraction formulas}: Directly
writing extraction formulas can be challenging for end-users, so Joker
provides a PBD interface that synthesizes formulas from user
demonstrations. A key aspect of our design is that the program
synthesized from the demonstration is made visible as a spreadsheet
formula that can be subsequently edited by the user, and is more easily
understood than imperative code due to its declarative form.

Section~\ref{sec:examples} describes a concrete scenario, showing how
Joker enables a user to complete a useful customization task. In
Section~\ref{sec:implementation}, we outline the implementation of our
formula language and user interface, as well as the algorithms used by
our PBD interface. We evaluate our approach in
Section~\ref{sec:evaluation} by describing a suite of case studies in
which we used Joker to extract and augment a variety of websites in
order to characterize its capabilities and limitations. Joker relates to
existing work not only in end-user web customization, but also in
end-user web scraping and program synthesis, which we discuss in
Section~\ref{sec:related-work}. Finally, we discuss opportunities for
future work in Section~\ref{sec:conclusion}.

\hypertarget{sec:examples}{%
\section{Example Usage Scenario}\label{sec:examples}}

Here is an example scenario, illustrated in Figure~\ref{fig:ebay}, and
demonstrated in the video accompanying this paper. Jen is searching for
a karaoke machine on eBay, a shopping website. She wants to use Joker to
sort products by price within a page of search results, a feature not
supported by eBay.

\begin{figure*}
  \includegraphics[width=\textwidth]{media/ebay.png}
  \caption{\label{fig:ebay}Scraping and customizing eBay by unified demonstration and formulas.}
\end{figure*}

\emph{Extracting Product Titles \& Prices By Demonstration}.
(Figure~\ref{fig:ebay} Part A): Jen initiates Joker through the browser
context menu. As she hovers over the product title, Joker provides two
kinds of live feedback. First, it highlights the titles of all the
corresponding product listings on the page, to indicate how it has
generalized Jen's intent based on her demonstration of a single example
product. Second, a column of the table is populated with the values that
will be extracted, giving a preview of how the extracted data would
look. To commit to this extraction, Jen clicks on the product title. She
repeats a similar process to extract the price.

When she clicks one of the cells in column \texttt{B}, the formula bar
above the table displays the extraction formula that was generated from
her demonstration:

\texttt{=QuerySelector(rowElement,\ "span.s-item\_\_price")}

This formula produces the values in column \texttt{B}. For each row in
the table, Joker has a created a reference to the corresponding DOM
element, accessible through the special identifier \texttt{rowElement}.
The \texttt{QuerySelector} function runs the specified CSS selector
(\texttt{span.s-item\_\_price}) within the row element to extract the
price of each item. While Jen could directly edit this formula, in this
case the values in the table are correct, so there's no need to edit the
formula.

\emph{Sorting Products By Price}. (Figure~\ref{fig:ebay} Part B): Next,
Jen wants to sort the table by price, but she realizes that the column
of prices is a list of strings containing the \texttt{\$} symbol. She
needs to \emph{augment} this raw data by turning the strings into
numbers. In the next column (\texttt{C}), Jen starts typing a formula,
and uses an autocomplete dropdown to find a relevant function that
extracts numeric values from strings:

\texttt{=ExtractNumber(B)}

Now column C contains numeric values, so Jen can sort the table by
price. Because the table is synchronized with the website, the product
listings become sorted as well.

\emph{Extracting Product URL With Formulas} (Figure~\ref{fig:ebay} Part
C): While completing the previous task, Jen realizes there is another
customization she would find useful: prioritizing products for which she
has not yet visited the details page for the listing. To do this, she
must first return to \emph{extracting} more relevant data from the page.

Each product listing links to a page for the specific product, but
because the URL is not visible in the page, it's not possible to
directly extract it by demonstration. However, Jen can still achieve the
goal by directly writing an extraction formula.

Jen has some basic knowledge of HTML, which she can leverage to write
the formula. She opens the browser developer tools, and observes that
the listing title is represented by a link tag with a heading inside:

\begin{verbatim}
<a href="LISTING PAGE URL">
  <h3>LISTING NAME</h3>
</a>
\end{verbatim}

Since there is already a column \texttt{A} in the table representing the
product's title, she can use this as a starting point to write the
formula:

\texttt{=GetAttribute(GetParent(A),\ "href")}

This formula first calls the \texttt{GetParent} function on column
\texttt{A}, traversing up a level from the
\texttt{\textless{}h3\textgreater{}} elements to the
\texttt{\textless{}a\textgreater{}} elements. Then,
\texttt{GetAttribute} extracts the \texttt{href} value containing the
link URL. After running this formula, the table contains a column
\texttt{D} with the URL for each product.

\emph{Sorting Products By Whether They Have Been Visited}
(Figure~\ref{fig:ebay} Part C): After performing that extraction, Jen
can write a final augmentation formula to indicate whether she has
visited the corresponding product page:

\texttt{=Visited(D)}

The \texttt{Visited} function checks whether a URL is present in the
browser history and returns a boolean value represented by a checkbox.
Jen can then sort by this column to put listings that she has not yet
visited at the top of the page.

Using Joker, Jen was able to not only achieve her initial customization
goal to sort the products by price but also perform a customization she
did not plan to do. This was made possible by Joker's unified
interaction model for web customization which enabled her to interleave
extraction and augmentation. While we have described only a single
illustrative scenario in this example, Joker is flexible enough to
support a wide range of other useful customizations and workflows on
various websites, described in more detail in
Section~\ref{sec:evaluation}.

\hypertarget{sec:implementation}{%
\section{System Implementation}\label{sec:implementation}}

\begin{figure*}
  \includegraphics[width=\textwidth]{media/overview.png}
  \caption{\label{fig:overview}An overview of Joker's interaction model and wrapper induction process.}
\end{figure*}

In this section, we describe Joker's formula language in more detail.
Then, we briefly outline the \emph{wrapper induction}
\citep{kushmerick2000} algorithm that Joker's PBD interface uses to
synthesize the row element and column selectors presented in formulas.

\hypertarget{extraction-formulas}{%
\subsection{Extraction Formulas}\label{extraction-formulas}}

The Wildcard customization tool includes a formula language for
augmentation, including operators for basic arithmetic and string
manipulation, as well as more advanced operators that fetch data from
web APIs. As in other tabular interfaces like SIEUFERD \citep{bakke2016}
and Airtable \citep{2021f}, formulas apply to a whole column at a time
rather than a single cell, and can reference other columns by name.
Joker extends this base language with new constructs which enable it to
apply to \emph{data extraction} instead of just augmentation.

We added DOM elements as a data type in the language, alongside strings,
numbers, and booleans. Because the language runs in a JavaScript
interpreter, we simply use native JavaScript values to represent DOM
elements in the language. DOM elements are displayed visually by showing
their inner text contents. They can also be implicitly typecast to
strings for use in other formulas; for example, a string manipulation
formula like \texttt{Substring} can be called on a DOM element value,
and will operate on its text contents.

We also added several functions to the formula language for traversing
the DOM and performing extractions, summarized below with their types:

\begin{itemize}
\tightlist
\item
  \texttt{QuerySelector(el:\ Element,\ sel:\ string):\ Element}.
  Executes the CSS selector \texttt{sel} inside of element \texttt{el},
  and returns the first matching element.
\item
  \texttt{GetAttribute(el:\ Element,\ attribute:\ string):\ string}.
  Returns the value for an attribute on an element.
\item
  \texttt{GetParent(el:\ Element):\ Element}. Returns the parent of a
  given element.
\end{itemize}

To extract data from a row, formulas need a way to reference the current
row, so we added a construct to support this use case. Every row in the
table maps to one DOM element in the page; we allow formulas to access
this DOM element via a special keyword, \texttt{rowElement}. In some
sense, \texttt{rowElement} can be seen as a hidden extra column of data
in the table containing DOM elements.

While many more functions could be added to expose more of the
underlying DOM API, we found that in practice these three functions
provided ample power through composition. For example, in
Section~\ref{sec:examples} we showed how \texttt{GetParent} and
\texttt{GetAttribute} can be composed to traverse the DOM and extract
the URL associated with a product listing.

By providing a single formula language to express extractions and
augmentations, Joker enables a \emph{unified interaction model} that
supports interleaving the two. Furthermore, the formula language enables
users to specify logic using pure, stateless functions that reactively
update in response to upstream changes. This \emph{functional reactive
paradigm} is easier to reason about than traditional imperative
programming, as demonstrated by the use of formulas by millions of end
users in spreadsheet programs and end-user programming environments
\citep{2021g, 2021h, 2021f, 2021a, 2021c, chang2014}.

\hypertarget{wrapper-induction}{%
\subsection{Wrapper Induction}\label{wrapper-induction}}

When users demonstrate a specific column value to extract, Joker must
synthesize a program that reflects the user's general intent. This is an
instance of the \emph{wrapper induction} problem of synthesizing a web
data extraction query from examples. Prior work on this topic
\citep{kushmerick2000, furche2016} prioritizes accuracy and robustness
to future changes, which makes sense for a fully automated system, but
can lead to very complex queries. In our work, we chose to prioritize
the readability of queries by less sophisticated users, so that users
can more easily author queries and repair them when they break. We
implemented a set of heuristics inspired by Vegemite \citep{lin2009} for
wrapper induction, illustrated in Figure~\ref{fig:overview}.

\hypertarget{sec:evaluation}{%
\section{Evaluation}\label{sec:evaluation}}

We evaluate our interaction model and tool by describing the results of
our use of Joker to extract data and perform customizations on popular
websites. For the websites on which Joker can be used, we provide the
sequence of interactions needed to achieve the customizations; for the
websites on which Joker fails, we explain the relevant limitations.

\hypertarget{successful-applications}{%
\subsection{Successful Applications}\label{successful-applications}}

We have used Joker to achieve a variety of customizations across many
websites. Table 1 summarizes examples we have found on popular websites.

\emph{Sorting search results by price on Amazon.} We have found Joker to
be useful for sorting the contents of various websites. One example of a
useful sort achieved by Joker is sorting search results by price within
the Featured page on Amazon. (Using Amazon's sort by price feature often
returns irrelevant results.) In Amazon's source code, the price is split
into three HTML elements: the dollar sign, the dollar amount, and the
cents amount. A user can extract by demonstration only the cents element
into column \texttt{A}. Subsequently, because the parent element of the
cents element contains all three of the price elements, the user can
extract the full price using the formula \texttt{GetParent(A)}. Next,
the user can write the formula \texttt{ExtractNumber(B)} to convert the
string into a numeric value. Finally, the user can sort this column by
low-to-high prices. In a similar manner, we have used Joker to extract
and sort prices and ratings on the product listing pages of Target and
eBay.

\emph{Filtering titles of publications on Google Scholar.} We have also
found Joker can be useful for filtering a website's listings based on
the text content of an element in the listing. For example, we have used
Joker to filter the titles of a researcher's publications on their
Google Scholar profile which is not natively supported. First, a user
can extract the titles into a column (\texttt{A}) by demonstration.
Then, the user can write the formula \texttt{Includes(A,\ "compiler")}
that returns whether or not the title contains the keyword ``compiler.''
Finally, the user can sort by this column to get all of the publications
that fit their constraint at the top of the page. We have also used
Joker to filter other text-based directory web pages such as Google
search results and the MIT course catalog, in similar ways.

\emph{Retrieving information about links on Reddit.} Additionally, we
have used Joker to augment web pages with external information. For
example, Joker can augment Reddit's user interface, which has a list of
headlines with links to articles, with the links' read times and whether
the link has already been read. To achieve this customization, a user
first extracts the headline elements into column (\texttt{A}) by
demonstration. The user can then extract the link into the next column
(\texttt{B}) with the formula \texttt{GetAttribute(A,\ "href")}. Then,
the user can write the formula \texttt{ReadTimeInSeconds(B)} that calls
an API that returns the links' read times. Similarly, the user can write
the formula \texttt{Visited(B)}, which uses another API that returns
whether that link has been visited in the user's browser history. The
user can also extract elements such as the number of comments and the
time of posting and sort by these values. We have performed similar
customizations on websites such as ABC News and CNN.

\hypertarget{limitations}{%
\subsection{Limitations}\label{limitations}}

Joker is most effective on websites whose data is presented as a
collection of similarly-structured HTML elements. Certain websites,
however, have designs that make it difficult for Joker to extract data:

\begin{itemize}
\tightlist
\item
  \emph{Heterogeneous row elements.} Some websites break their content
  into rows, but the rows do not have a consistent layout, and contain
  different types of child elements. For example, the page design of
  HackerNews alternates between rows containing a title and rows
  containing supplementary data (e.g.~number of likes and the time of
  posting). Because Joker only chooses a single row selector, when
  extracting by demonstration, Joker will only select one of the types
  of rows, and elements in the other types of rows will not be
  extracted.
\item
  \emph{Infinite scroll.} Some websites have an ``infinite scroll''
  feature that adds new entries to the page when a user scrolls to the
  bottom. Joker's table will only contain elements that were rendered
  when the table was first created. Additionally, for websites that
  render a very large number of DOM elements, the speed of the live
  feedback provided by Joker's PBD interface might significantly
  decrease. This is because the wrapper induction process used by the
  PBD interface queries the DOM which takes longer as the size of the
  DOM increases.
\end{itemize}

\begin{table*}[]
\centering
\begin{tabular}{|l|l|}
\hline
\textbf{Website}              & \textbf{Example Customization Achieved by Joker}                                        \\ \hline
eBay, Amazon            & Filter listings by whether they have a "Sponsored" label.                        \\
Amazon, Target & Sort search results by price and rating.                                                \\
Google Scholar                & Filter publications for those whose title contains a user-provided keyword. \\
Reddit, CNN, ABC  & Sort by the read times of articles. Filter already-visited articles.         \\
Weather.com                   & Filter hourly weather to find sunny times of day.                                        \\
Github                        & Sort a user's code repositories by stars to find popular work.                          \\
Postmates, Uber Eats    & Sort restaurants by delivery time and delivery fee.                                     \\ \hline
\end{tabular}
\vspace{8pt}
\caption{Examples of websites that Joker can be used to customize, including extraction and augmentation}
\end{table*}

\hypertarget{sec:related-work}{%
\section{Related Work}\label{sec:related-work}}

\hypertarget{end-user-web-customization}{%
\subsection{End-user Web
Customization}\label{end-user-web-customization}}

Joker builds on web customization ideas implemented by previous tools.
Our contribution is a new interaction model that allows for interleaving
extraction and augmentation, enabled by a unified formula language and a
PBD interface.

Joker is an extension of the Wildcard customization system
\citep{litt2020}, and preserves its foundational idea of synchronizing a
table with a website. Wildcard only allows for extraction logic to be
written by programmers in Javascript; our work has substantially
extended the Wildcard formula language and added an entire new system
for dynamically creating data extraction logic within the user
interface. We also improved the formula editing interface by adding an
autocomplete dropdown and documentation popup, which proved important in
our testing for allowing end-users to reliably edit and create formulas.

Vegemite \citep{lin2009} is a tool for end-user programming of web
mashups. Like Joker, it allows users to perform demonstrations to
extract data, but Vegemite only displays a table after all the
demonstrations have been provided, which rules out interleaving
extraction and augmentation. Vegemite does allow users to directly view
and edit some of the logic generalized from demonstrations, but it only
allows for editing augmentation logic, not extraction logic. The wrapper
induction algorithm used in Joker is also very similar to Vegemite's
algorithm.

Sifter \citep{huynh2006} is a tool that augments websites with advanced
sorting and filtering functionality. It attempts to automatically detect
items and fields on the website with a variety of heuristics. If these
fail, it gives the user the option of demonstrating to correct some
parts of the result. In contrast, Joker makes fewer assumptions about
the structure of websites, by giving control to the user from the
beginning of the process and displaying an editable synthesized program.

\hypertarget{end-user-web-scraping-and-program-synthesis}{%
\subsection{End-user Web Scraping and Program
Synthesis}\label{end-user-web-scraping-and-program-synthesis}}

Joker builds on insights from other tools that synthesize web scraping
(i.e.~data extraction) code from user demonstrations, and give users
ways to inspect and modify the generated code.

Rousillon \citep{chasins2018} is a tool that enables end-users to
extract hierarchical web data across multiple linked web pages. It
presents the web extraction program generated from demonstrations in an
editable, high-level, block-based language called Helena \citep{2021c}.
While both Rousillon and Joker create an editable program, they have
different focuses. Because Rousillon allows users to extract data across
multiple pages (e.g., extracting details from each linked page in a
list), it uses an imperative language, with nested loops as a key
construct. In contrast, Joker can only extract within a single page, and
therefore can use a simpler declarative formula language. Also,
Rousillon only allows editing high-level control flow and treats some
details of the extraction logic as opaque; Joker offers finer-grained
control over details like CSS selectors.

Mayer et al propose a user interaction model called \emph{Program
Navigation} \citep{mayer2015} which aims to give users another mechanism
beside examples for guiding the generalization process of PBE tools like
FlashExtract \citep{le2014} and FlashFill \citep{harris}. This is
important because demonstrations are an ambiguous specification for
program synthesis \citep{peleg2018}: the set of synthesized programs for
a demonstration can be very large. Joker shares the general idea of
displaying synthesized programs, but only presents the top-ranked
program.

More broadly, Joker's use of PBD to generate editable code embodies Ravi
Chugh's notion of \emph{prodirect manipulation} \citep{chugh2016a},
implemented in Sketch-N-Sketch \citep{chugh2016}, which aims to bridge
the divide between programmatic and direct manipulation.

\hypertarget{sec:conclusion}{%
\section{Conclusion And Future Work}\label{sec:conclusion}}

In this paper, we presented a unified interaction model for web
customization. Our key idea is a spreadsheet formula language that
encompasses both extraction and augmentation tasks, along with a
programming-by-demonstration (PBD) interface that makes it easy for
end-users to program to create formulas. The main area of future work
involves making the formula language more accessible to end-users not
familiar with CSS selectors. Our ultimate goal is to enable anyone that
uses the web to customize websites in the course of their daily use in
an intuitive and flexible way.

% \printbibliography

%%
%% The next two lines define the bibliography style to be used, and
%% the bibliography file.
\bibliographystyle{ACM-Reference-Format}
\bibliography{references-bibtex.bib}

\end{document}
\endinput
%%
%% End of file `sample-sigconf.tex'.
